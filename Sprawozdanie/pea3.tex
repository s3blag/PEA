\documentclass{article}
\usepackage[table,xcdraw]{xcolor} % Dla tabel
\usepackage{siunitx} % Provides the \SI{}{} and \si{} command for typesetting SI units
\usepackage{graphicx} % Required for the inclusion of images
\usepackage{natbib} % Required to change bibliography style to APA
\usepackage{amsmath} % Required for some math elements 
\setlength\parindent{12pt} % Removes all indentation from paragraphs
%\usepackage{times} % Uncomment to use the Times New Roman font
\usepackage{amsmath}
\usepackage[export]{adjustbox}
\usepackage{tikz}

\usepackage{pgfplots}
\usepackage[utf8]{inputenc} % Język polski
\usepackage{polski}
\usepackage[polish]{babel}

\usepackage[top=1in, bottom=1.25in, left=1.25in, right=1.25in]{geometry} % Marginesy
\usepackage{listings} % Kod programu
\usepackage{indentfirst} % Wcięcie przy pomocy \par
\usepackage{multicol} % Kilka kolumn dla itemize
\usepackage{color} % Kolorowanie tekstu
\usepackage{float}

%Projektowanie Efektywnych Algorytmów \\ {\normalsize Implementacja i analiza efektywności metody podziału i ograniczeń dla problemu komiwojażera}


%\renewcommand{\labelenumi}{\alph{enumi}.} % Zamienia litery w enumerate na a, b, c, ...
%----------------------------------------------------------------------------------------
%	DOCUMENT INFORMATION
%----------------------------------------------------------------------------------------
\title{Projektowanie Efektywnych Algorytmów \\ {\normalsize Implementacja i analiza efektywności metody Tabu Search dla problemu komiwojażera}}
\author{Sebastian Łągiewski 226173\\Łukasz Zatorski 226172 }

% Title page layout (fold)
\makeatletter
\renewcommand{\maketitle}{
	\begin{titlepage}
		\begin{center}
			\vspace*{3cm}
			\LARGE \@title \par
			\vspace{2cm}
			\textit{\small Autorzy:}\par
			\normalsize \@author\par \normalsize
			\vspace{3cm}
			Prowadzący : mgr inż. Radosław Idzikowski\\
			\vspace{3cm}
			Wydział Elektroniki\\ III rok \par
			\vspace{3cm}
			\small \@date
		\end{center}
	\end{titlepage}
}
\makeatother
\begin{document}
	\maketitle % Insert the title, author and date

%\begin{center}
%\begin{tabular}{l r}
%Data: & Semestr letni, 2017 \\ % Date the experiment was performed

%Prowadzący:  mgr inż. Radosław Idzikowski% Instructor/supervisor
%\end{tabular}
%\end{center}
\tableofcontents


% If you wish to include an abstract, uncomment the lines below
% \begin{abstract}
% Abstract text
% \end{abstract}

%----------------------------------------------------------------------------------------
%	SECTION 0
%----------------------------------------------------------------------------------------

%\newpage
%\section{Tytuł}} 
%\subsection{Tytuł}
%\par Tekst

%\begin{itemize}
%	\item\textit{Tekst} - Opis
%\end{itemize}

%\begin{enumerate}
%	\item Opis
%\end{enumerate}

%\begin{center}
%	\includegraphics[scale=0.6, center]{Screenshot_1.png}
%\end{center}

%\begin{lstlisting}[basicstyle=\small]
%	kod programu;
%\end{lstlisting}

%----------------------------------------------------------------------------------------
%	SECTION 1
%----------------------------------------------------------------------------------------
\newpage
\section{Opis algorytmu}
\par Algorytm genetyczny to rodzaj heurystyki przeszukującej przestrzeń  rozwiązań problemu w celu wyszukania rozwiązania optymalnego. Problem definiuje środowisko, w którym istnieje pewna populacja osobników. Każdy z osobników ma przypisany pewien zbiór informacji stanowiących jego genotyp, czyli konkretne rozwiązanie.
\newline
\newline
Pierwszym wymaganiem algorytmu genetycznego jest utworzenie populacji początkowej (losowego zbioru rozwiązań). Następnie N razy odbywa się selekcja, gdzie N oznacza rozmiar populacji rodzicielskiej - zaimplementowano wersję turniejową. W każdym turnieju bierze udział Q osobników, gdzie Q - rozmiar turnieju. W wyniku takiej selekcji otrzymujemy populację rodzicielską - grupę osobników, które będą poddawane krzyżowaniu - w zaimplementowanym algorytmi użyto operatora OX. Po skrzyżowaniu, każdy osobnik z określonym prawdopodobieństwem Pm może podlegać mutacji, czyli losowemu zdeformowaniu rozwiązania. W opisywanym algorytmie zaimplementowano dwie metody krzyżowania - swap oraz invert.
Warunkiem zakończenia algorytmu jest upływ określonego czasu.




%----------------------------------------------------------------------------------------
%	SECTION 3
%----------------------------------------------------------------------------------------

%----------------------------------------------------------------------------------------
%	SECTION 4
%----------------------------------------------------------------------------------------

%----------------------------------------------------------------------------------------
%	BIBLIOGRAPHY
%----------------------------------------------------------------------------------------
%\newpage
%\bibliohystyle{apalike}
%\begin{thebibliography}{9}
%
%\bibitem{wikipediacluster} 
%Wikipedia: Computer cluster,
%\\\texttt{https://en.wikipedia.org/wiki/Computer\_cluster}
%	
%\bibitem{wikipediasieve} 
%Wikipedia: Sieve of Erastosthenes,
%\\\texttt{https://en.wikipedia.org/wiki/Sieve\_of\_Eratosthenes}
%
%\end{thebibliography}

%----------------------------------------------------------------------------------------
\end{document}
